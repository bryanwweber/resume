% arara: xelatex
% arara: xelatex
%%%%%%%%%%%%%%%%%%%%%%%%%%%%%%%%%%%%%%%%%%%%%%%%%%%%%%%%%%%%%%%%%%%%%%%%
%%%%%%%%%%%%%%%%%%%%%% Simple LaTeX CV Template %%%%%%%%%%%%%%%%%%%%%%%%
%%%%%%%%%%%%%%%%%%%%%%%%%%%%%%%%%%%%%%%%%%%%%%%%%%%%%%%%%%%%%%%%%%%%%%%%

%%%%%%%%%%%%%%%%%%%%%%%%%%%%%%%%%%%%%%%%%%%%%%%%%%%%%%%%%%%%%%%%%%%%%%%%
%% NOTE: If you find that it says                                     %%
%%                                                                    %%
%%                           1 of ??                                  %%
%%                                                                    %%
%% at the bottom of your first page, this means that the AUX file     %%
%% was not available when you ran LaTeX on this source. Simply RERUN  %%
%% LaTeX to get the ``??'' replaced with the number of the last page  %%
%% of the document. The AUX file will be generated on the first run   %%
%% of LaTeX and used on the second run to fill in all of the          %%
%% references.                                                        %%
%%%%%%%%%%%%%%%%%%%%%%%%%%%%%%%%%%%%%%%%%%%%%%%%%%%%%%%%%%%%%%%%%%%%%%%%

%%%%%%%%%%%%%%%%%%%%%%%%%%%% Document Setup %%%%%%%%%%%%%%%%%%%%%%%%%%%%

% Don't like 10pt? Try 11pt or 12pt
\documentclass[11pt,final]{article}
\usepackage{fontspec}
\usepackage{varwidth}
\usepackage{microtype}
%
%This document must be compiled with XeLaTeX. This means that any font
%installed on the system can be used. Select the font here.
%
\defaultfontfeatures{Mapping=tex-text}
% \setmainfont[Path = fonts/ ,
%              Ligatures = TeX ,
%              Extension = .otf ,
%              BoldFont = Andada-Bold ,
%              ItalicFont = Andada-Italic ,
%              SmallCapsFont = AndadaSC-Regular
% ]
% {Andada-Regular}
\usepackage[default]{raleway}
\newfontface\head[Path=fonts/]{GenBkBasB.ttf}
\newfontface\sectionfont[Path=fonts/,Scale=0.9]{PlayfairDisplaySC-Regular.ttf}

\newcommand*{\org}[1]{\textit{#1}}
\newcommand*{\position}[1]{\textbf{#1}}
\newcommand*{\meauthor}[1]{\textbf{#1}}
\newcommand*{\papertitle}[1]{\textit{#1}}

% Set the text to justified without hyphenation
\tolerance=1
\emergencystretch=\maxdimen
\hyphenpenalty=10000
\hbadness=10000

% This is a helpful package that puts math inside length specifications
\usepackage{calc}

% This package helps LaTeX auto-hyphenate hyphenated words if you use
% special hyphens. For example, bio\-/mimicry will properly hyphenate
% ``mimicry'' if necessary.
\usepackage[shortcuts]{extdash}

% Layout: Puts the section titles on left side of page
\reversemarginpar

%
%         PAPER SIZE, PAGE NUMBER, AND DOCUMENT LAYOUT NOTES:
%
% The next \usepackage line changes the layout for CV style section
% headings as marginal notes. It also sets up the paper size as either
% letter or A4. By default, letter was used. If A4 paper is desired,
% comment out the letterpaper lines and uncomment the a4paper lines.
%
% As you can see, the margin widths and section title widths can be
% easily adjusted.
%
% ALSO: Notice that the includefoot option can be commented OUT in order
% to put the PAGE NUMBER *IN* the bottom margin. This will make the
% effective text area larger.
%
% IF YOU WISH TO REMOVE THE ``of LASTPAGE'' next to each page number,
% see the note about the +LP and -LP lines below. Comment out the +LP
% and uncomment the -LP.
%
% IF YOU WISH TO REMOVE PAGE NUMBERS, be sure that the includefoot line
% is uncommented and ALSO uncomment the \pagestyle{empty} a few lines
% below.
%

%% Use these lines for letter-sized paper
\usepackage[paper=letterpaper,
            %includefoot, % Uncomment to put page number above margin
            marginparwidth=1.3in,     % Length of section titles
            marginparsep=.025in,       % Space between titles and text
            bmargin=1in,               % 1 inch margins
            tmargin=0.75in,
            lmargin=0.5in,
            rmargin=0.75in,
            includemp]{geometry}

%% Use these lines for A4-sized paper
%\usepackage[paper=a4paper,
%            %includefoot, % Uncomment to put page number above margin
%            marginparwidth=30.5mm,    % Length of section titles
%            marginparsep=1.5mm,       % Space between titles and text
%            margin=25mm,              % 25mm margins
%            includemp]{geometry}

%% More layout: Get rid of indenting throughout entire document
\setlength{\parindent}{0in}

% Provides special list environments and macros to create new ones
\usepackage[shortlabels]{enumitem}

% Allow reverse numbered lists
\usepackage{etaremune}

% Simpler bibsections for CV sections
% (thanks to natbib for inspiration)
%
% * For lists of references with hanging indents and no numbers:
%
%   \begin{bibsection}
%       \item ...
%   \end{bibsection}
%
% * For numbered lists of references (with hanging indents):
%
%   \begin{bibenum}
%       \item ...
%   \end{bibenum}
%
%   Note that bibenum numbers continuously throughout. To reset the
%   counter, use
%
%   \restartlist{bibenum}
%
%   at the place where you want the numbering to reset.

\makeatletter
\newlength{\bibhang}
\setlength{\bibhang}{1em}
\newlength{\bibsep}
 {\@listi \global\bibsep\itemsep \global\advance\bibsep by\parsep}
\newlist{bibsection}{itemize}{3}
\setlist[bibsection]{label=,leftmargin=\bibhang,%
        itemindent=-\bibhang,
        itemsep=\bibsep,parsep=\z@,partopsep=0pt,
        topsep=0pt}
\newlist{bibenum}{enumerate}{3}
\setlist[bibenum]{label=[\arabic*],resume,leftmargin={\bibhang+\widthof{[999]}},%
        itemindent=-\bibhang,
        itemsep=\bibsep,parsep=\z@,partopsep=0pt,
        topsep=0pt}

% Add a new environment for reversed numbered bibliography lists
\newenvironment{bibmune}
{\renewcommand\labelenumi{[\theenumi]}%
\etaremune[
 topsep=0pt,
 itemsep=\bibsep,
 parsep=0pt,partopsep=0pt,
 itemindent=0pt,
 leftmargin={\widthof{[999]}}]}
{\endetaremune}

\let\oldendbibenum\endbibenum
\def\endbibenum{\oldendbibenum\vspace{-.6\baselineskip}}
\let\oldendbibsection\endbibsection
\def\endbibsection{\oldendbibsection\vspace{-.6\baselineskip}}
\makeatother

%% Reference the last page in the page number
%
% NOTE: comment the +LP line and uncomment the -LP line to have page
%       numbers without the ``of ##'' last page reference)
%
% NOTE: uncomment the \pagestyle{empty} line to get rid of all page
%       numbers (make sure includefoot is commented out above)
%
\usepackage{datetime}
\newdateformat{mydate}{\monthname[\THEMONTH] \THEYEAR}
\usepackage{fancyhdr,lastpage}
\pagestyle{fancy}
%\pagestyle{empty}      % Uncomment this to get rid of page numbers
\fancyhf{}\renewcommand{\headrulewidth}{0pt}
\fancyfootoffset{\marginparsep+\marginparwidth}
\newlength{\footpageshift}
\setlength{\footpageshift}
          {\marginparsep+\marginparwidth}
\lfoot{\hspace{\footpageshift}%
       \parbox{\textwidth}{\line(1,0){428.2}\\%
       \, \mydate\today \hfill%
%                    \arabic{page} of \protect\pageref*{LastPage} % +LP
                    Bryan W.\ Weber \quad \arabic{page}% -LP
                    }}%

% Finally, give us PDF bookmarks
\usepackage{color,hyperref}
\definecolor{darkblue}{rgb}{0.0,0.0,0.3}
\hypersetup{
    colorlinks,
    breaklinks,
    linkcolor=darkblue,
    urlcolor=darkblue,
    anchorcolor=darkblue,
    citecolor=darkblue,
    pdfinfo={
        Title={Curriculum Vitae of Bryan W. Weber},
        Author={Bryan W. Weber},
    },
}

%%%%%%%%%%%%%%%%%%%%%%%% End Document Setup %%%%%%%%%%%%%%%%%%%%%%%%%%%%


%%%%%%%%%%%%%%%%%%%%%%%%%%% Helper Commands %%%%%%%%%%%%%%%%%%%%%%%%%%%%

%%% HEADING AT TOP OF CURRICULUM VITAE

% The title (name) with a horizontal rule under it
% (optional argument typesets an object right-justified across from name
%  as well)
%
% Usage: \makeheading{name}
%        OR
%        \makeheading[right_object]{name}
%
% Place at top of document. It should be the first thing.
% If ``right_object'' is provided in the square-braced optional
% argument, it will be right justified on the same line as ``name'' at
% the top of the CV. For example:
%
%       \makeheading[\emph{Curriculum vitae}]{Your Name}
%
% will put an emphasized ``Curriculum vitae'' at the top of the document
% as a title. Likewise, a picture could be included:
%
%   \makeheading[{\includegraphics[height=1.5in]{my_picture}}]{Your Name}
%
% the picture will be flush right across from the name. For this example
% to work, make sure the extra set of curly braces is included. Also
% makes sure that \usepackage{graphicx} is somewhere in the preamble.
\newcommand{\makeheading}[2][]%
        {\hspace*{-\marginparsep minus \marginparwidth}%
         \begin{minipage}[t]{\textwidth+\marginparwidth+\marginparsep}%
             {\huge #2 \hfill #1}\\[-0.15\baselineskip]%
                 \rule{\columnwidth}{1pt}%
         \end{minipage}}

%%% SECTION HEADINGS

% The section headings. Flush left in small caps down pseudo-margin.
%
% Usage: \section{section name}
\renewcommand{\section}[1]{\pagebreak[3]%
    \vspace{1.3\baselineskip}%
    \phantomsection\addcontentsline{toc}{section}{#1}%
    \noindent\llap{\smash{\parbox[t]{\marginparwidth}{\hyphenpenalty=10000\raggedright #1}}}%
    \vspace{-\baselineskip}\par}

%%% LISTS

% This macro alters a list by removing some of the space that follows the list
% (is used by lists below)
\newcommand*\fixendlist[1]{%
    \expandafter\let\csname preFixEndListend#1\expandafter\endcsname\csname end#1\endcsname
    \expandafter\def\csname end#1\endcsname{\csname preFixEndListend#1\endcsname\vspace{-0.6\baselineskip}}}

% These macros help ensure that items in outer-type lists do not get
% separated from the next line by a page break
% (they are used by lists below)
\let\originalItem\item
\newcommand*\fixouterlist[1]{%
    \expandafter\let\csname preFixOuterList#1\expandafter\endcsname\csname #1\endcsname
    \expandafter\def\csname #1\endcsname{\let\oldItem\item\def\item{\pagebreak[2]\oldItem}\csname preFixOuterList#1\endcsname}
    \expandafter\let\csname preFixOuterListend#1\expandafter\endcsname\csname end#1\endcsname
    \expandafter\def\csname end#1\endcsname{\let\item\oldItem\csname preFixOuterListend#1\endcsname}}
\newcommand*\fixinnerlist[1]{%
    \expandafter\let\csname preFixInnerList#1\expandafter\endcsname\csname #1\endcsname
    \expandafter\def\csname #1\endcsname{\let\oldItem\item\let\item\originalItem\csname preFixInnerList#1\endcsname}
    \expandafter\let\csname preFixInnerListend#1\expandafter\endcsname\csname end#1\endcsname
    \expandafter\def\csname end#1\endcsname{\csname preFixInnerListend#1\endcsname\let\item\oldItem}}

% An itemize-style list with lots of space between items
%
% Usage:
%   \begin{outerlist}
%       \item ...    % (or \item[] for no bullet)
%   \end{outerlist}
\newlist{outerlist}{itemize}{3}
    \setlist[outerlist]{label=\enskip\textbullet,leftmargin=*}
    \fixendlist{outerlist}
    \fixouterlist{outerlist}

% An environment IDENTICAL to outerlist that has better pre-list spacing
% when used as the first thing in a \section
%
% Usage:
%   \begin{lonelist}
%       \item ...    % (or \item[] for no bullet)
%   \end{lonelist}
\newlist{lonelist}{itemize}{3}
    \setlist[lonelist]{label=\enskip\textbullet,leftmargin=*,partopsep=0pt,topsep=0pt}
    \fixendlist{lonelist}
    \fixouterlist{lonelist}

% An itemize-style list with little space between items
%
% Usage:
%   \begin{innerlist}
%       \item ...    % (or \item[] for no bullet)
%   \end{innerlist}
\newlist{innerlist}{itemize}{3}
    \setlist[innerlist]{label=\enskip\textbullet,leftmargin=*,parsep=0pt,itemsep=0pt,topsep=0pt,partopsep=0pt}
    \fixinnerlist{innerlist}

% An environment IDENTICAL to innerlist that has better pre-list spacing
% when used as the first thing in a \section
%
% Usage:
%   \begin{loneinnerlist}
%       \item ...    % (or \item[] for no bullet)
%   \end{loneinnerlist}
\newlist{loneinnerlist}{itemize}{3}
    \setlist[loneinnerlist]{label=\enskip\textbullet,leftmargin=*,parsep=0pt,itemsep=0pt,topsep=0pt,partopsep=0pt}
    \fixendlist{loneinnerlist}
    \fixinnerlist{loneinnerlist}

%%% EXTRA SPACE

% To add some paragraph space between lines.
% This also tells LaTeX to preferably break a page on one of these gaps
% if there is a needed pagebreak nearby.
\newcommand{\blankline}{\quad\pagebreak[3]}
\newcommand{\halfblankline}{\quad\vspace{-0.5\baselineskip}\pagebreak[3]}

%%% FORMATTING MACROS

% Provides a linked \url{#1} that doesn't require escape characters
\usepackage{url}

\RequirePackage{academicons}
\RequirePackage{fontawesome}
\newcommand{\iconspace}{\enspace}
\newcommand{\arxiv}[1]{\aiarXiv\iconspace arxiv:\href{https://arxiv.org/abs/#1}{#1}}
\newcommand{\doi}[1]{\aiDoi\iconspace doi:\href{https://doi.org/#1}{#1}}
\newcommand{\figshare}[1]{\aiFigshare\iconspace figshare:\href{https://doi.org/#1}{#1}}
\newcommand{\refurl}[1]{\faGlobe\iconspace\url{#1}}

% You can adjust the style \url{} uses here:
% (options are: same, rm, sf, tt; defaults to tt)
\urlstyle{same}

% For \email{ADDRESS}, links ADDRESS to the url mailto:ADDRESS
% (uncomment to typeset the e\-/mail address in typewriter font;
%  otherwise, will be typeset in the \urlstyle above)
\DeclareUrlCommand\emaillink{\urlstyle{same}}
\providecommand*\emaillink[1]{\nolinkurl{#1}}
\providecommand*\email[1]{\href{mailto:#1}{\emaillink{#1}}}

\providecommand\BibTeX{{B\kern-.05em{\sc i\kern-.025em b}\kern-.08em \TeX}}
\providecommand\Matlab{\textsc{Matlab}}

% Custom hyphenation rules for words that LaTeX has trouble with
\hyphenation{bio-mim-ic-ry bio-in-spi-ra-tion re-us-a-ble pro-vid-er Media-Wiki fuels}

%%%%%%%%%%%%%%%%%%%%%%%% End Helper Commands %%%%%%%%%%%%%%%%%%%%%%%%%%%

%----------------------------------------------------------------------%
% The following is copyright and licensing information for
% redistribution of this LaTeX source code; it also includes a liability
% statement. If this source code is not being redistributed to others,
% it may be omitted. It has no effect on the function of the above code.
%----------------------------------------------------------------------%
% Copyright (c) 2007, 2008, 2009, 2010, 2011 by Theodore P. Pavlic
%
% Unless otherwise expressly stated, this work is licensed under the
% Creative Commons Attribution-Noncommercial 3.0 United States License. To
% view a copy of this license, visit
% http://creativecommons.org/licenses/by-nc/3.0/us/ or send a letter to
% Creative Commons, 171 Second Street, Suite 300, San Francisco,
% California, 94105, USA.
%
% THE SOFTWARE IS PROVIDED "AS IS", WITHOUT WARRANTY OF ANY KIND, EXPRESS
% OR IMPLIED, INCLUDING BUT NOT LIMITED TO THE WARRANTIES OF
% MERCHANTABILITY, FITNESS FOR A PARTICULAR PURPOSE AND NONINFRINGEMENT.
% IN NO EVENT SHALL THE AUTHORS OR COPYRIGHT HOLDERS BE LIABLE FOR ANY
% CLAIM, DAMAGES OR OTHER LIABILITY, WHETHER IN AN ACTION OF CONTRACT,
% TORT OR OTHERWISE, ARISING FROM, OUT OF OR IN CONNECTION WITH THE
% SOFTWARE OR THE USE OR OTHER DEALINGS IN THE SOFTWARE.
%----------------------------------------------------------------------%

%
%This document must be compiled with XeLaTeX. This means that any font
%installed on the system can be used. Select the font here.
%
\defaultfontfeatures{Mapping=tex-text}
\setmainfont[Path = fonts/ ,
             Ligatures = TeX ,
             Extension = .otf ,
             BoldFont = Andada-Bold ,
             ItalicFont = Andada-Italic ,
             SmallCapsFont = AndadaSC-Regular
]
{Andada-Regular}
\newfontface\head[Path=fonts/]{GenBkBasB.ttf}
\newfontface\sectionfont[Path=fonts/,Scale=0.9]{PlayfairDisplaySC-Regular.ttf}

%%%%%%%%%%%%%%%%%%%%%%%%% Begin CV Document %%%%%%%%%%%%%%%%%%%%%%%%%%%%

\begin{document}
\vspace{1em}
\makeheading{{\head Bryan W.\ Weber}}

\section{{\sectionfont Contact Information}}

% MACROS: \rcolwidth is the width of the right column of the table
%             (adjust it to your liking; default is 1.85in).
%         \ccolwidth is width of area between left and right boxes.
%
\newlength{\rcolwidth}
\setlength{\rcolwidth}{2.5in}%
\newlength{\ccolwidth}
\setlength{\ccolwidth}{1pt}
\newlength{\lcolwidth}
\setlength{\lcolwidth}{\textwidth-\rcolwidth-\ccolwidth}
%
% Address box
\begin{varwidth}{\lcolwidth}%
Department of Mechanical Engineering\\
University of Connecticut\\
191 Auditorium Road U-3139\\
Storrs, CT 06269 USA%
\end{varwidth}%
\hfill
%Vertical rule separating address from other contact info.
\begin{varwidth}{\ccolwidth}
\rule{0.5pt}{4\baselineskip} %Comment this line to remove the line.
\end{varwidth}%
\hfill
% Non-snail-mail contact information
\begin{varwidth}{\rcolwidth}%
\textit{E-mail:} \email{weber@engr.uconn.edu}\\
\textit{Work:} +1-860-486-2590 \\
\textit{Cell:} +1-412-443-6447 \\
\textit{Web:} \href{http://bryanwweber.com}{bryanwweber.com}
\end{varwidth}%

\section{{\sectionfont Research Interests}}

My research interests generally lie in developing tools to apply fundamental
combustion insights to solve engineering problems. My recent work
involves developing experimental methods to analyze intermediate species
at practical combustion conditions. I am also interested in developing
methods to analyze computational models, particularly kinetic models of
combustion.

\section{{\sectionfont Education}}

Ph.D., Mechanical Engineering, University of Connecticut, 2014

M.S., Mechanical Engineering, University of Connecticut, 2010

B.S.E., Aerospace Engineering, Case Western Reserve University, 2009

\section{{\sectionfont Professional Experience}}

Visiting Assistant Professor, University of Connecticut \hfill 2014--Present

Conducting research on the combustion kinetics of alternative and traditional
fuels. Teaching undergraduate courses in thermal-fluids engineering.

\vspace{\baselineskip}

Graduate Research Assistant, University of Connecticut \hfill 2009--2014\\
Undergraduate Research Assistant, Case Western Reserve University \hfill 2007--2009\\
\href{http://combdiaglab.engr.uconn.edu}{Combustion Diagnostics Laboratory} --- Director: C.-J.\ Sung

Conducted experimental and computational studies of the ignition
properties of several alternative fuels and foundational fuels, with
focus on engine-relevant conditions.
Designed a species sampling apparatus for time-resolved
species measurements in the rapid compression machine, using gas
chromatography/mass spectrometry to identify and quantify
combustion intermediates.
Analyzed kinetic models for combustion to determine the parameters
controlling prediction of ignition delay and improve the ability of
the models to predict combustion events.

\section{{\sectionfont Journal Publications}}

\begin{bibmune}
\item \textbf{B.W.\ Weber}, M.W.\ Renfro, and C.J.\ Sung. \textit{On
        the Uncertainty of Temperature Estimation in a Rapid
        Compression Machine.} Under review at Combustion and Flame,
        Jan.\ 2015.

\item S.M.\ Burke, U.\ Burke, R.\ Mc Donagh, O.\ Mathieu, I.\ Osorio,
        C.\ Keesee, A.\ Morones, E.L.\ Petersen, W.\ Wang, T.A.\ DeVerter,
        M.A.\ Oehlschlaeger, B.\ Rhodes, R.K.\ Hanson, D.F.\ Davidson,
        \textbf{B.W.\ Weber}, C.J.\ Sung, J.\ Santner, Y.\ Ju, F.M.\ Haas,
        F.L.\ Dryer, E.N.\ Volkov, E.J.\ Nilsson, A.A.\ Konnov, M.\ Alrefae,
        F.\ Khaled, A.\ Farooq, P.\ Dirrenberger, P.A.\ Glaude,
        F.\ Battin-Leclerc, and H.J.\ Curran.\ \textit{An Experimental and
        Modeling Study of Propene Oxidation. Part 2: Ignition Delay Time
        and Flame Speed Measurements.} Combustion and Flame, vol.\ 162, no.\ 2,
        pp.\ 296--314, Feb.\ 2015. \\ \doi{10.1016/j.combustflame.2014.07.032}

\item \textbf{B.W.\ Weber}, W.J.\ Pitz, M.\ Mehl, A.C.\ Davis,
        E.J.\ Silke, and C.J.\ Sung. \textit{Experiments and
        Modeling of the Autoignition of Methylcyclohexane at High
        Pressure.} Combustion and Flame, vol.\ 161, no.\ 8, pp.\
        1972--1983, Aug.\ 2014.
        \doi{10.1016/j.combustflame.2014.01.018}

\item S.M.\ Sarathy, S.\ Park, \textbf{B.W.\ Weber}, W.\ Wang,
        P.S.\ Veloo, A.C.\ Davis, C.\ Togbé, C.K.\ Westbrook, O.\ Park,
        G.\ Dayma, Z.\ Luo, M.A.\ Oehlschlaeger, F.N.\ Egolfopoulos,
        T.\ Lu, W.J.\ Pitz, C.J.\ Sung, and P.\ Dagaut. \textit{A
        Comprehensive Experimental and Modeling Study of iso-Pentanol
        Combustion.} Combustion and Flame, vol.\ 160, no.\ 12, pp.\
        2712--2728, Dec.\ 2013. \doi{10.1016/j.combustflame.2013.06.022}

\item \textbf{B.W.\ Weber} and C.J.\ Sung. \textit{Comparative
        Autoignition Trends in Butanol Isomers at Elevated Pressure.}
        Energy and Fuels, vol.\ 27, no.\ 3, pp.\ 1688--1698, Mar.\ 2013. \\
        \doi{10.1021/ef302195c}

\item T.\ Tsujimura, W.J.\ Pitz, F.\ Gillespie, H.J.\ Curran,
        \textbf{B.W.\ Weber}, Y.\ Zhang, and C.J.\ Sung.
        \textit{Development of Isopentanol Reaction Mechanism
        Reproducing Autoignition Character at High and Low
        Temperatures.} Energy and Fuels, vol.\ 26, no.\ 8, pp.\ 4871--4886,
        Aug.\ 2012. \doi{10.1021/ef300879k}

\item \textbf{B.W.\ Weber}, K.\ Kumar, Y.\ Zhang, and C.J.\ Sung.
        \textit{Autoignition of n-butanol at elevated pressure and
        low-to-intermediate temperature.} Combustion and Flame,
        vol.\ 158, no.\ 5, pp.\ 809--819, Mar.\ 2011.
        \doi{10.1016/j.combustflame.2011.02.005}
\end{bibmune}

% Add a little space to nudge next ``Conference Publications'' marginpar
% down to make room for tall ``Submitted Conference Publications''
% marginpar. If there are enough submitted journal publications, this
% space will not be needed (and should be removed).
%\vspace{0.1in}

\section{{\sectionfont Conference Publications and Presentations}}

\begin{bibmune}
\item S.S.\ Merchant (Presenting), W.H.\ Green, K.M.\ Van Geem, N.
        Hansen, \textbf{B.W.\ Weber}, and C.J.\ Sung.
        \textit{Combustion of the Butanol Isomers: Reaction Pathways
        from High to Low Temperature.} \ordinalnum{8} International
        Conference on Chemical Kinetics, University Seville, Seville,
        Spain, Jul.\ 2013.

\item \textbf{B.W.\ Weber}, W.J.\ Pitz, C.J.\ Sung, M.\ Mehl,
        E.J.\ Silke, and A.C.\ Davis. \textit{Experiments and Modeling of
        the Autoignition of Methyl-Cyclohexane at High Pressure.}
        Paper 3A02, \ordinalnum{8} US National Technical Meeting of the
        Combustion Institute, Park City, UT, May 2013.

\item \textbf{B.W.\ Weber}, S.S.\ Merchant, C.J.\ Sung, and W.H.\ Green.
        \textit{An Autoignition Study of iso-Butanol: Experiments and
        Modeling.} Paper 3A01, \ordinalnum{8} US National Technical
        Meeting of the Combustion Institute, Park City, UT, May 2013.

\item S.M.\ Sarathy, S.\ Park, W.\ Wang, P.\ Veloo, A.C.\ Davis,
        C.\ Togbé, \textbf{B.W.\ Weber}, C.K.\ Westbrook, O.\ Park,
        G.\ Dayma, Z.\ Luo, M.A.\ Oehlschlaeger, F.\ Egolfopoulos, T.\ Lu,
        W.J.\ Pitz, C.J.\ Sung, and P.\ Dagaut. \textit{A Comprehensive
        Experimental and Modeling Study of iso-Pentanol Combustion.}
        Paper 2A12, \ordinalnum{8} US National Technical Meeting of the
        Combustion Institute, Park City, UT, May 2013.

\item \textbf{B.W.\ Weber} and C.J.\ Sung. \textit{Comparative
        Investigation of the High Pressure Autoignition of the Butanol
        Isomers.} Paper A-01, Fall Technical Meeting of the Eastern
        States Section of the Combustion Institute, Storrs, CT, Oct.\
        2011.

\item M.R.\ Harper, W.H.\ Green (Presenting), K.M.\ Van Geem,
        \textbf{B.W.\ Weber}, C.J.\ Sung, I.\ Stranic, D.F.\ Davidson,
        and R.K.\ Hanson. \textit{Combustion of the butanol isomers:
        Reaction pathways at elevated pressures from low-to-high
        temperatures.} Paper \#84, \ordinalnum{7} International
        Conference on Chemical Kinetics, Massachusetts Institute of
        Technology, Cambridge, MA, Jul.\ 2011.

\item \textbf{B.W.\ Weber} and C.J.\ Sung. \textit{A Rapid Compression
        Study of the Butanol Isomers at Elevated Pressure.} Paper 1B13,
        \ordinalnum{7}  US National Technical Meeting of the Combustion
        Institute, Georgia Institute of Technology, Atlanta, GA, Mar.\
        2011.

\item \textbf{B.W.\ Weber}, K.\ Kumar, and C.J.\ Sung.
        \textit{Autoignition of Butanol Isomers at Low to Intermediate
        Temperature and Elevated Pressure.} Paper AIAA-2011-0316,
        \ordinalnum{49}  Annual Aerospace Sciences Meeting,
        Orlando, FL, Jan.\ 2011.
\end{bibmune}

\section{{\sectionfont Conference Posters}}

\begin{bibmune}
\item \textbf{B.W.\ Weber} and C.J.\ Sung. \textit{Validation of
        Kinetic Models of the Butanol Isomers At High Pressure
        using a Rapid Compression Machine.} Poster T40,
        \ordinalnum{7} International Conference on Chemical Kinetics,
         Massachusetts Institute of Technology, Cambridge, MA,
         Jul.\ 2011.

\item \textbf{B.W.\ Weber}. \textit{Autoignition of n-Butanol at
        Elevated Pressure and Low to Intermediate Temperature.}
        \ordinalnum{1} Combustion Energy Frontier Research Center
        Annual Meeting, Princeton University, Princeton, NJ,
        Sep.\ 2010.

\item \textbf{B.W.\ Weber}, K.\ Kumar, and C.J.\ Sung. \textit{An
        Investigation of Hydrocarbon Flames using Probe Sampling and
        Gas Chromatography/Mass Spectrometry.} Support of Undergraduate
        Research and Creative Endeavors Symposium and Poster Session,
        Case Western Reserve University, Cleveland, OH, Apr.\ 2009.
\end{bibmune}

\section{{\sectionfont Other Presentations}}

\begin{bibmune}
\item \textbf{ B.W.\ Weber} and C.J.\ Sung. \textit{Analysis of
        Hydrocarbon Fuels using Gas Chromatography/Mass Spectrometry.}
        Summer Undergraduate Research in Energy Sciences Program,
        Dominion Energy East Ohio Branch, Cleveland, OH, Aug.\ 2008.
\end{bibmune}

\vspace{0.1in}

\section{{\sectionfont Teaching Experience}}

\textbf{University of Connecticut, Storrs, CT, USA}

\vspace{\baselineskip}
Fall 2014 --- Fundamentals of Engineering Thermodynamics

Spring 2013 --- Introduction to Mechanical Engineering (Instructor of Record)

\section{{\sectionfont Professional Service}}

Combustion Energy Frontier Research Center (CEFRC) \hfill 2012--2014\\
Lead Chair, Junior Associates Committee

\begin{innerlist}
\item[] Coordinate monthly teleconferences for graduate students
and post-doctoral researchers in the CEFRC where junior members of
the CEFRC present recent research results to the group.
Act as the liaison between the Center's principle investigators
and the junior members.
\end{innerlist}

\vspace{\baselineskip}

U.S.\ Department of Energy \hfill 2013--2014\\
Member, EFRC Newsletter Editorial Board

\begin{innerlist}
\item[] Contribute articles to the Energy Frontier Research Centers (EFRC)
newsletter describing recent scientific advances resulting from
EFRC research, including:\\
\href{http://www.energyfrontier.us/newsletter/201210/burning-butanol-better-engine}
{``Burning Butanol in a Better Engine''}\\
\href{http://www.energyfrontier.us/newsletter/201401/advantage-renewable-fuels-high-efficiency-engines}
{``The Advantage of Renewable Fuels in High-Efficiency Engines''}\\
\href{http://www.energyfrontier.us/newsletter/201404/confined-catalysts-last-longer}
{``Confined Catalysts Last Longer''}\\
The audience for the articles is scientifically literate, but not
expert in the fields relevant to each article. Edit articles written
by other board members for factual and grammatical correctness.
\end{innerlist}

\vspace{\baselineskip}

Journal Referee
   \begin{innerlist}
       \item[] Energy \& Fuels
       \item[] Proceedings of the Combustion Institute
       \item[] Combustion Science \& Technology
       \item[] SAE World Congress
   \end{innerlist}

\vspace{0.5em}

\section{{\sectionfont Awards and Fellowships}}

Doctoral Dissertation Fellowship, University of Connecticut, 2014

Graduate Predoctoral Fellowship, Department of Mechanical
Engineering, 2013

Graduate Teaching Fellowship, Department of Mechanical Engineering, 2013

Graduate Assistantship in Areas of National
Need, University of Connecticut, 2010

Fred H.\ Vose Prize, Department of Mechanical and
Aerospace Engineering, 2009

Summer Undergraduate Research in Energy Sciences
Grant, Case Western Reserve University, 2008

\section{{\sectionfont Professional Memberships}}
AIAA - Member

ASME - Member

The Combustion Institute - Member

ACS - Member

\end{document}

%%%%%%%%%%%%%%%%%%%%%%%%%% End CV Document %%%%%%%%%%%%%%%%%%%%%%%%%%%%%
