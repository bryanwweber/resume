%%%%%%%%%%%%%%%%%%%%%%%%%%%%%%%%%%%%%%
%
%     TITLE NAME
%
%%%%%%%%%%%%%%%%%%%%%%%%%%%%%%%%%%%%%%
\namesection{Bryan Weber}{\urlstyle{same}\href{https://bryanwweber.com}{bryanwweber.com} |
\href{mailto:bryan.w.weber@gmail.com}{bryan.w.weber@gmail.com} | \href{https://github.com/bryanwweber}{\faGithub/bryanwweber} | Active TS/SCI Clearance
}
\sectionsep
\section{Summary}

\raggedright

% Engineer with 10+ years of problem solving experience working on open source development.
% Seeking an opportunity to impact the next-generation data stack by strategically applying emerging technologies in novel domains as a senior software engineer.
Engineer with 10+ years of problem solving experience, seeking an opportunity to impact mission outcomes by strategically applying emerging technologies to build complex, data-driven solutions.
% Experienced software engineer seeking to support mission outcomes across product lines by leveraging expertise in building complex, data-driven solutions.
Proven ability to collaborate effectively, own the software development lifecycle, and drive innovation to deliver results for stakeholders.
% Proven ability to collaborate effectively, own projects end-to-end, and drive innovation to deliver impactful results for stakeholders.
Passionate about building cohesive teams through servant leadership, supporting teammates, and fostering a culture of continuous process improvement.

\sectionsep

%%%%%%%%%%%%%%%%%%%%%%%%%%%%%%%%%%%%%%
%     EXPERIENCE
%%%%%%%%%%%%%%%%%%%%%%%%%%%%%%%%%%%%%%
% Integrated repo with pantsbuild, contributing taplo and pyproject.toml features upstream
% Presented three lunch-and-learn topics for the engineering org covering advanced Python topics: Mise-en-place, Python packaging, Pantsbuild
% Defined core team objectives for the ML-focused team
% Deployed Slack application to notify users of relevant threads
% Developed workflows using Flyte to train and deploy ML models to Torchserve
% Developed and deployed an application agnostic HTTP API for ML model inference
% Supported continuous model improvement functionality for integrated software tracker (the mail box)
\setlist{itemsep=0.07ex, leftmargin=0.6cm}
\section{Experience}
\runsubsection{Staff Software Engineer}\location{| March 2024 – June 2024}\\
\runsubsection{Senior Software Engineer}\location{| August 2022 – March 2024}\\
\descript{Rebellion Defense}; growth-stage defense tech startup valued over \$1B.

\begin{itemize}
\item Lead Python architect on a modular and scalable microservice-based architecture to process tens-to-hundreds of thousands of sensor messages for battlefield observability. Built with Python, golang, Redis, PostgreSQL, gRPC, and RabbitMQ. Collaborated closely with product management and senior engineering leadership to deliver strategic features for high-profile customer focused demonstrations. Ensured on-time delivery with consistent, open, and honest discussion of status and risk mitigation strategies.
\item Led as the subject matter expert for ML-based task assignment algorithms during \$1M/year-contract close-out demonstration to the Chief Scientist of the Air Force.
\item Led 1-month integration of novel ML algorithms using ONNX and PyTorch for a \$1M/year contract. Achieved the product goal of evaluating thousands of potential actions per second, while current capabilities are limited to dozens per day.
% \item Integrated LLM-generated summaries of unstructured data from action reports in our adversary emulation software. Provided a technical report describing trade offs of several LLM integration frameworks in terms of cost, maintenance impact, and ease of use.
\item Developed distributed systems that efficiently deployed ML models to perform cyber-asset criticality assessment and automated training data generation. Used Flyte, Kubernetes, and AWS EKS to maintain model provenance.
% \item Championed productivity by seeking out and implementing more efficient development tools while strategically removing unused ones. Examples include integrating a consistent build system to three repositories working with approximately 20 engineers. Contributed in-house features back upstream to the open-source project.
% \item Founding member of a cross-disciplinary team focused on cultivating an inspiring and inclusive professional environment through initiatives promoting empathy, coaching and mentorship, and open communication.
\item Reviewed hundreds of code changes from dozens of team members to support agile-based workflows. Emphasized collaborative development, driving customer value, and motivating team growth through shared skill development.
% \item Collaborated with product, sales, and other stakeholders to submit proposals worth >\$1M to Federal requests for proposals (RFPs).
\end{itemize}
\sectionsep

\runsubsection{Core Developer and Steering Committee Member}\location{| January 2014–Present}\\
\descript{Cantera Project | \normalfont\href{https://github.com/Cantera}{\faGithub/Cantera}} ; open-source C++/Python library for system-level chemical simulations

% \vspace{\topsep}
\begin{itemize}
% \item Cantera is the largest open-source code for thermodynamics, chemical kinetics, and mass transport calculations
\item Migrated continuous integration infrastructure for automated tests and packaging from Travis CI and Appveyor to GitHub Actions
\item Developed and automated build of Conda and PyPI packages for our C-extension using GitHub Actions, downloaded 500,000+ times in 5+ years.
\item Maintained the extensive unit, integration, and regression test suite to ensure reliability for scientific users.
% \item Maintained SCons-based build system for macOS, Linux, and Windows, supporting GCC, Clang, MSVC, and MinGW compilers
% \item Developed and automated build and deployment of the Cantera website to Linode hosting using GitHub Actions, saving hundreds of hours of developer time.
% \item Received \$2.5M grant from NSF to expand Cantera to novel scientific domains and develop sustainable communities.
% \item Managed the 2020 community survey, receiving over 60 responses and leading to dropping support for Win32
% \item Moderated Cantera User's Group, responding to 1,200+ posts over 10+ years. Focused on cultivating an inspiring and inclusive professional environment by demonstrating empathy, coaching and mentorship, and open communication.
% \item Organized and led three training workshops at international conferences with up to 100 paying attendees per workshop. Generated over \$10,000 of revenue for the project.
% \item Mentored 2 GSoC students under the NumFOCUS umbrella, leading to two feature enhancements
\end{itemize}
\sectionsep

\runsubsection{Open Source Software Engineer}\location{| January 2022 – May 2022}\\
\descript{Coiled Computing}; early stage SaaS startup making cloud management tools integrated with Dask.

% Worked with GB-TB scale Parquet files to find and fix performance bottlenecks with cloud (S3) storage
% Merged 15 PRs in 4 months
% Refactored DataFrame creation APIs to use high-level graphs, enabling performance optimizations
% Developed Coiled Cloud widget for v2
% Participated in community team issue triage, question answering
\begin{itemize}
\item Led performance improvement feature for Dask to enable loading terabyte-scale Parquet-formatted data from cloud storage. Results led to changing recommendations for settings in Dask to prevent out-of-memory errors.
\item Led development of a highly-requested dashboard to display cluster status in the Coiled client.
% \item Optimized task graph generation in Dask by improving three Dask DataFrame APIs
% \item Added hooks for \verb|dask.distributed.Client| to run user-supplied preload plugins
\end{itemize}
\sectionsep

\runsubsection{Director of Undergraduate Studies}\location{| August 2019 – January 2022}\\
\runsubsection{Assistant Professor in Residence}\location{| August 2014 – January 2022}\\
\descript{Mechanical Engineering Department, University of Connecticut}

% \vspace{\topsep}
\begin{itemize}
\item Managed deployment and maintenance of JupyterHub to on-premises RedHat virtual machine via Docker and Docker Compose, scaled for use by 200+ students per semester. Enabled courses to have consistent software base and significantly reduced student issues due to software installation.
% \item Produced 150 lecture videos, viewed over 218,000 times over three years, with 1,400+ YouTube channel subscribers.
\item Led 2-year development of new Mechanical Engineering curriculum for over 800 undergraduate students, balancing the needs of students, faculty, and industrial partners. Incorporated modern computation for engineering throughout the curriculum for the first time at UConn.
\item Provided guidance and mentoring to 10+ undergraduate researchers working on open-source software. All the students had code changes merged in prominent open-source projects for scientific computing.
% \item Manged 4--10 teaching assistants per year, mentoring them in communication and teaching skills.
% \item Supported department administrative functions, including hiring staff and student communications
\item Research effectiveness of using Jupyter Notebooks in engineering course work for assignments and projects, published in two conference proceedings
% \item Taught 200+ undergraduate students each semester, achieving median 5/5 rating on student evaluations
\end{itemize}
\sectionsep

\runsubsection{Author}\location{2021–Present}\\
\descript{\normalfont\urlstyle{same}\href{https://orbital-mechanics.space}{orbital-mechanics.space}}
\begin{itemize}
\item Teaching resource developed for a course at the University of Connecticut.
\item Currently among the top 10 results Google search results for the query \verb|orbital mechanics|, \verb|hohmann transfer|, and related terms.
\end{itemize}
\sectionsep

%%%%%%%%%%%%%%%%%%%%%%%%%%%%%%%%%%%%%%
%     SERVICE
%%%%%%%%%%%%%%%%%%%%%%%%%%%%%%%%%%%%%%

% \section{Service}
\runsubsection{Co-Chair, Small Development Grants Committee}\location{| January 2019 – January 2023}\\
\descript{NumFOCUS}
\begin{itemize}
\item Awarded up to \$95,000 three times annually to applicants from among NumFOCUS sponsored and affiliated projects. Evaluated up to 40 applications per round.
\item Organized and co-hosted decision meetings for 13 committee members, ensuring each application was discussed in the allotted time and each committee member was contributing effectively.
\item Provided thoughtful, actionable, feedback to projects that were not selected for funding. Focused on empathetic communication to ensure projects were able to respond effectively in future funding rounds.
\end{itemize}
\sectionsep

\runsubsection{Freelance Author and Technical Reviewer}\location{| January 2019 – January 2021}\\
\descript{Real Python | \normalfont\href{https://realpython.com/team/bweber/}{realpython.com}}
\begin{itemize}
\item Wrote six in-depth articles averaging over 30,000 unique readers per week
\item Covered basic to advanced topics in the Python ecosystem, including \href{https://realpython.com/python-main-function/}{Python \texttt{main} functions}, \href{https://realpython.com/python-datetime/}{using \texttt{datetime} and \texttt{dateutil}}, and \href{https://realpython.com/python-enumerate/}{the \texttt{enumerate} function}.
\item Featured on the \href{https://pythonbytes.fm/episodes/show/151/certified-it-works-on-my-machine}{Python Bytes} and \href{https://realpython.com/podcasts/rpp/21/}{Real Python} podcasts.
\end{itemize}
% \sectionsep

%%%%%%%%%%%%%%%%%%%%%%%%%%%%%%%%%%%%%%
%     EDUCATION
%%%%%%%%%%%%%%%%%%%%%%%%%%%%%%%%%%%%%%

\section{Education}

\descript{Ph.D.\ in Mechanical Engineering,}
\location{University of Connecticut | 2014}

% \descript{M.S.\ in Mechanical Engineering}
% \location{University of Connecticut | 2010}

\descript{B.S.E.\ in Aerospace Engineering,}
\location{Case Western Reserve University | 2009}
\sectionsep

%%%%%%%%%%%%%%%%%%%%%%%%%%%%%%%%%%%%%%
%     PROJECTS
%%%%%%%%%%%%%%%%%%%%%%%%%%%%%%%%%%%%%%

% \section{Other Select Projects}
% \begin{minipage}[t]{0.23\textwidth}
% \textbf{\href{https://github.com/orgs/conda-forge/teams?query=\%40bryanwweber}{conda-forge}}\\
% Maintainer of 11 conda-forge packages
% \end{minipage}
% \hfill\vline\hfill
% \begin{minipage}[t]{0.23\textwidth}
% \textbf{\href{https://github.com/bryanwweber/thermostate}{ThermoState}}\\
% Used in thermodynamics classes and published in the Journal of Open Source Education
% \end{minipage}
% \hfill\vline\hfill
% \begin{minipage}[t]{0.23\textwidth}
% \textbf{\href{https://github.com/bryanwweber/MyST-NB-Bokeh}{MyST-NB-Bokeh}}\\
% Compute Bokeh figures and paste them into JupyterBook pages
% \end{minipage}
% % \hfill\vline\hfill
% % \begin{minipage}[t]{0.23\textwidth}
% % \textbf{\href{https://github.com/bryanwweber/cansen}{CanSen}}\\
% % Simplified input-file based user interface for Cantera
% % \end{minipage}
% \hfill\vline\hfill
% \begin{minipage}[t]{0.23\textwidth}
% \textbf{\href{https://github.com/bryanwweber/thermohw}{thermohw}}\\
% Converts Jupyter Notebooks into homework assignments via a custom \verb|nbconvert| template
% \end{minipage}

\section{Skills}

\textbf{Programming Languages/Packages}: Advanced Python\inlinespace Basic Go\inlinespace Basic C++\inlinespace NumPy\inlinespace pandas\inlinespace pytest\inlinespace FastAPI\inlinespace Pydantic \\[6pt]

\textbf{DevOps \& Infrastructure}: Docker\inlinespace Kubernetes\inlinespace CI/CD\inlinespace AWS Cloud Platforms (EC2, S3, EKS)\inlinespace Machine Learning Pipeline Development (Flyte, Dagster) \inlinespace Git, GitFlow, GitHub, GitLab \\[6pt]

\textbf{Software Development}: Software Design\inlinespace Software Architecture\inlinespace Project Management\inlinespace Team Leadership\inlinespace Code Review\inlinespace Communication
