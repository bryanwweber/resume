%%%%%%%%%%%%%%%%%%%%%%%%%%%%%%%%%%%%%%%%%%%%%%%%%%%%%%%%%%%%%%%%%%%%%%%%
%%%%%%%%%%%%%%%%%%%%%% Simple LaTeX CV Template %%%%%%%%%%%%%%%%%%%%%%%%
%%%%%%%%%%%%%%%%%%%%%%%%%%%%%%%%%%%%%%%%%%%%%%%%%%%%%%%%%%%%%%%%%%%%%%%%

%%%%%%%%%%%%%%%%%%%%%%%%%%%%%%%%%%%%%%%%%%%%%%%%%%%%%%%%%%%%%%%%%%%%%%%%
%% NOTE: If you find that it says                                     %%
%%                                                                    %%
%%                           1 of ??                                  %%
%%                                                                    %%
%% at the bottom of your first page, this means that the AUX file     %%
%% was not available when you ran LaTeX on this source. Simply RERUN  %%
%% LaTeX to get the ``??'' replaced with the number of the last page  %%
%% of the document. The AUX file will be generated on the first run   %%
%% of LaTeX and used on the second run to fill in all of the          %%
%% references.                                                        %%
%%%%%%%%%%%%%%%%%%%%%%%%%%%%%%%%%%%%%%%%%%%%%%%%%%%%%%%%%%%%%%%%%%%%%%%%

%%%%%%%%%%%%%%%%%%%%%%%%%%%% Document Setup %%%%%%%%%%%%%%%%%%%%%%%%%%%%

% Don't like 10pt? Try 11pt or 12pt
\documentclass[11pt,final]{article}
\usepackage{fontspec}
\usepackage{varwidth}
\usepackage{microtype}
%
%This document must be compiled with XeLaTeX. This means that any font
%installed on the system can be used. Select the font here.
%
\defaultfontfeatures{Mapping=tex-text}
% \setmainfont[Path = fonts/ ,
%              Ligatures = TeX ,
%              Extension = .otf ,
%              BoldFont = Andada-Bold ,
%              ItalicFont = Andada-Italic ,
%              SmallCapsFont = AndadaSC-Regular
% ]
% {Andada-Regular}
\usepackage[default]{raleway}
\newfontface\head[Path=fonts/]{GenBkBasB.ttf}
\newfontface\sectionfont[Path=fonts/,Scale=0.9]{PlayfairDisplaySC-Regular.ttf}

\newcommand*{\org}[1]{\textit{#1}}
\newcommand*{\position}[1]{\textbf{#1}}
\newcommand*{\meauthor}[1]{\textbf{#1}}
\newcommand*{\papertitle}[1]{\textit{#1}}

% Set the text to justified without hyphenation
\tolerance=1
\emergencystretch=\maxdimen
\hyphenpenalty=10000
\hbadness=10000

% This is a helpful package that puts math inside length specifications
\usepackage{calc}

% This package helps LaTeX auto-hyphenate hyphenated words if you use
% special hyphens. For example, bio\-/mimicry will properly hyphenate
% ``mimicry'' if necessary.
\usepackage[shortcuts]{extdash}

% Layout: Puts the section titles on left side of page
\reversemarginpar

%
%         PAPER SIZE, PAGE NUMBER, AND DOCUMENT LAYOUT NOTES:
%
% The next \usepackage line changes the layout for CV style section
% headings as marginal notes. It also sets up the paper size as either
% letter or A4. By default, letter was used. If A4 paper is desired,
% comment out the letterpaper lines and uncomment the a4paper lines.
%
% As you can see, the margin widths and section title widths can be
% easily adjusted.
%
% ALSO: Notice that the includefoot option can be commented OUT in order
% to put the PAGE NUMBER *IN* the bottom margin. This will make the
% effective text area larger.
%
% IF YOU WISH TO REMOVE THE ``of LASTPAGE'' next to each page number,
% see the note about the +LP and -LP lines below. Comment out the +LP
% and uncomment the -LP.
%
% IF YOU WISH TO REMOVE PAGE NUMBERS, be sure that the includefoot line
% is uncommented and ALSO uncomment the \pagestyle{empty} a few lines
% below.
%

%% Use these lines for letter-sized paper
\usepackage[paper=letterpaper,
            %includefoot, % Uncomment to put page number above margin
            marginparwidth=1.3in,     % Length of section titles
            marginparsep=.025in,       % Space between titles and text
            bmargin=1in,               % 1 inch margins
            tmargin=0.75in,
            lmargin=0.5in,
            rmargin=0.75in,
            includemp]{geometry}

%% Use these lines for A4-sized paper
%\usepackage[paper=a4paper,
%            %includefoot, % Uncomment to put page number above margin
%            marginparwidth=30.5mm,    % Length of section titles
%            marginparsep=1.5mm,       % Space between titles and text
%            margin=25mm,              % 25mm margins
%            includemp]{geometry}

%% More layout: Get rid of indenting throughout entire document
\setlength{\parindent}{0in}

% Provides special list environments and macros to create new ones
\usepackage[shortlabels]{enumitem}

% Allow reverse numbered lists
\usepackage{etaremune}

% Simpler bibsections for CV sections
% (thanks to natbib for inspiration)
%
% * For lists of references with hanging indents and no numbers:
%
%   \begin{bibsection}
%       \item ...
%   \end{bibsection}
%
% * For numbered lists of references (with hanging indents):
%
%   \begin{bibenum}
%       \item ...
%   \end{bibenum}
%
%   Note that bibenum numbers continuously throughout. To reset the
%   counter, use
%
%   \restartlist{bibenum}
%
%   at the place where you want the numbering to reset.

\makeatletter
\newlength{\bibhang}
\setlength{\bibhang}{1em}
\newlength{\bibsep}
 {\@listi \global\bibsep\itemsep \global\advance\bibsep by\parsep}
\newlist{bibsection}{itemize}{3}
\setlist[bibsection]{label=,leftmargin=\bibhang,%
        itemindent=-\bibhang,
        itemsep=\bibsep,parsep=\z@,partopsep=0pt,
        topsep=0pt}
\newlist{bibenum}{enumerate}{3}
\setlist[bibenum]{label=[\arabic*],resume,leftmargin={\bibhang+\widthof{[999]}},%
        itemindent=-\bibhang,
        itemsep=\bibsep,parsep=\z@,partopsep=0pt,
        topsep=0pt}

% Add a new environment for reversed numbered bibliography lists
\newenvironment{bibmune}
{\renewcommand\labelenumi{[\theenumi]}%
\etaremune[
 topsep=0pt,
 itemsep=\bibsep,
 parsep=0pt,partopsep=0pt,
 itemindent=0pt,
 leftmargin={\widthof{[999]}}]}
{\endetaremune}

\let\oldendbibenum\endbibenum
\def\endbibenum{\oldendbibenum\vspace{-.6\baselineskip}}
\let\oldendbibsection\endbibsection
\def\endbibsection{\oldendbibsection\vspace{-.6\baselineskip}}
\makeatother

%% Reference the last page in the page number
%
% NOTE: comment the +LP line and uncomment the -LP line to have page
%       numbers without the ``of ##'' last page reference)
%
% NOTE: uncomment the \pagestyle{empty} line to get rid of all page
%       numbers (make sure includefoot is commented out above)
%
\usepackage{datetime}
\newdateformat{mydate}{\monthname[\THEMONTH] \THEYEAR}
\usepackage{fancyhdr,lastpage}
\pagestyle{fancy}
%\pagestyle{empty}      % Uncomment this to get rid of page numbers
\fancyhf{}\renewcommand{\headrulewidth}{0pt}
\fancyfootoffset{\marginparsep+\marginparwidth}
\newlength{\footpageshift}
\setlength{\footpageshift}
          {\marginparsep+\marginparwidth}
\lfoot{\hspace{\footpageshift}%
       \parbox{\textwidth}{\line(1,0){428.2}\\%
       \, \mydate\today \hfill%
%                    \arabic{page} of \protect\pageref*{LastPage} % +LP
                    Bryan W.\ Weber \quad \arabic{page}% -LP
                    }}%

% Finally, give us PDF bookmarks
\usepackage{color,hyperref}
\definecolor{darkblue}{rgb}{0.0,0.0,0.3}
\hypersetup{
    colorlinks,
    breaklinks,
    linkcolor=darkblue,
    urlcolor=darkblue,
    anchorcolor=darkblue,
    citecolor=darkblue,
    pdfinfo={
        Title={Curriculum Vitae of Bryan W. Weber},
        Author={Bryan W. Weber},
    },
}

%%%%%%%%%%%%%%%%%%%%%%%% End Document Setup %%%%%%%%%%%%%%%%%%%%%%%%%%%%


%%%%%%%%%%%%%%%%%%%%%%%%%%% Helper Commands %%%%%%%%%%%%%%%%%%%%%%%%%%%%

%%% HEADING AT TOP OF CURRICULUM VITAE

% The title (name) with a horizontal rule under it
% (optional argument typesets an object right-justified across from name
%  as well)
%
% Usage: \makeheading{name}
%        OR
%        \makeheading[right_object]{name}
%
% Place at top of document. It should be the first thing.
% If ``right_object'' is provided in the square-braced optional
% argument, it will be right justified on the same line as ``name'' at
% the top of the CV. For example:
%
%       \makeheading[\emph{Curriculum vitae}]{Your Name}
%
% will put an emphasized ``Curriculum vitae'' at the top of the document
% as a title. Likewise, a picture could be included:
%
%   \makeheading[{\includegraphics[height=1.5in]{my_picture}}]{Your Name}
%
% the picture will be flush right across from the name. For this example
% to work, make sure the extra set of curly braces is included. Also
% makes sure that \usepackage{graphicx} is somewhere in the preamble.
\newcommand{\makeheading}[2][]%
        {\hspace*{-\marginparsep minus \marginparwidth}%
         \begin{minipage}[t]{\textwidth+\marginparwidth+\marginparsep}%
             {\huge #2 \hfill #1}\\[-0.15\baselineskip]%
                 \rule{\columnwidth}{1pt}%
         \end{minipage}}

%%% SECTION HEADINGS

% The section headings. Flush left in small caps down pseudo-margin.
%
% Usage: \section{section name}
\renewcommand{\section}[1]{\pagebreak[3]%
    \vspace{1.3\baselineskip}%
    \phantomsection\addcontentsline{toc}{section}{#1}%
    \noindent\llap{\smash{\parbox[t]{\marginparwidth}{\hyphenpenalty=10000\raggedright #1}}}%
    \vspace{-\baselineskip}\par}

%%% LISTS

% This macro alters a list by removing some of the space that follows the list
% (is used by lists below)
\newcommand*\fixendlist[1]{%
    \expandafter\let\csname preFixEndListend#1\expandafter\endcsname\csname end#1\endcsname
    \expandafter\def\csname end#1\endcsname{\csname preFixEndListend#1\endcsname\vspace{-0.6\baselineskip}}}

% These macros help ensure that items in outer-type lists do not get
% separated from the next line by a page break
% (they are used by lists below)
\let\originalItem\item
\newcommand*\fixouterlist[1]{%
    \expandafter\let\csname preFixOuterList#1\expandafter\endcsname\csname #1\endcsname
    \expandafter\def\csname #1\endcsname{\let\oldItem\item\def\item{\pagebreak[2]\oldItem}\csname preFixOuterList#1\endcsname}
    \expandafter\let\csname preFixOuterListend#1\expandafter\endcsname\csname end#1\endcsname
    \expandafter\def\csname end#1\endcsname{\let\item\oldItem\csname preFixOuterListend#1\endcsname}}
\newcommand*\fixinnerlist[1]{%
    \expandafter\let\csname preFixInnerList#1\expandafter\endcsname\csname #1\endcsname
    \expandafter\def\csname #1\endcsname{\let\oldItem\item\let\item\originalItem\csname preFixInnerList#1\endcsname}
    \expandafter\let\csname preFixInnerListend#1\expandafter\endcsname\csname end#1\endcsname
    \expandafter\def\csname end#1\endcsname{\csname preFixInnerListend#1\endcsname\let\item\oldItem}}

% An itemize-style list with lots of space between items
%
% Usage:
%   \begin{outerlist}
%       \item ...    % (or \item[] for no bullet)
%   \end{outerlist}
\newlist{outerlist}{itemize}{3}
    \setlist[outerlist]{label=\enskip\textbullet,leftmargin=*}
    \fixendlist{outerlist}
    \fixouterlist{outerlist}

% An environment IDENTICAL to outerlist that has better pre-list spacing
% when used as the first thing in a \section
%
% Usage:
%   \begin{lonelist}
%       \item ...    % (or \item[] for no bullet)
%   \end{lonelist}
\newlist{lonelist}{itemize}{3}
    \setlist[lonelist]{label=\enskip\textbullet,leftmargin=*,partopsep=0pt,topsep=0pt}
    \fixendlist{lonelist}
    \fixouterlist{lonelist}

% An itemize-style list with little space between items
%
% Usage:
%   \begin{innerlist}
%       \item ...    % (or \item[] for no bullet)
%   \end{innerlist}
\newlist{innerlist}{itemize}{3}
    \setlist[innerlist]{label=\enskip\textbullet,leftmargin=*,parsep=0pt,itemsep=0pt,topsep=0pt,partopsep=0pt}
    \fixinnerlist{innerlist}

% An environment IDENTICAL to innerlist that has better pre-list spacing
% when used as the first thing in a \section
%
% Usage:
%   \begin{loneinnerlist}
%       \item ...    % (or \item[] for no bullet)
%   \end{loneinnerlist}
\newlist{loneinnerlist}{itemize}{3}
    \setlist[loneinnerlist]{label=\enskip\textbullet,leftmargin=*,parsep=0pt,itemsep=0pt,topsep=0pt,partopsep=0pt}
    \fixendlist{loneinnerlist}
    \fixinnerlist{loneinnerlist}

%%% EXTRA SPACE

% To add some paragraph space between lines.
% This also tells LaTeX to preferably break a page on one of these gaps
% if there is a needed pagebreak nearby.
\newcommand{\blankline}{\quad\pagebreak[3]}
\newcommand{\halfblankline}{\quad\vspace{-0.5\baselineskip}\pagebreak[3]}

%%% FORMATTING MACROS

% Provides a linked \url{#1} that doesn't require escape characters
\usepackage{url}

\RequirePackage{academicons}
\RequirePackage{fontawesome}
\newcommand{\iconspace}{\enspace}
\newcommand{\arxiv}[1]{\aiarXiv\iconspace arxiv:\href{https://arxiv.org/abs/#1}{#1}}
\newcommand{\doi}[1]{\aiDoi\iconspace doi:\href{https://doi.org/#1}{#1}}
\newcommand{\figshare}[1]{\aiFigshare\iconspace figshare:\href{https://doi.org/#1}{#1}}
\newcommand{\refurl}[1]{\faGlobe\iconspace\url{#1}}

% You can adjust the style \url{} uses here:
% (options are: same, rm, sf, tt; defaults to tt)
\urlstyle{same}

% For \email{ADDRESS}, links ADDRESS to the url mailto:ADDRESS
% (uncomment to typeset the e\-/mail address in typewriter font;
%  otherwise, will be typeset in the \urlstyle above)
\DeclareUrlCommand\emaillink{\urlstyle{same}}
\providecommand*\emaillink[1]{\nolinkurl{#1}}
\providecommand*\email[1]{\href{mailto:#1}{\emaillink{#1}}}

\providecommand\BibTeX{{B\kern-.05em{\sc i\kern-.025em b}\kern-.08em \TeX}}
\providecommand\Matlab{\textsc{Matlab}}

% Custom hyphenation rules for words that LaTeX has trouble with
\hyphenation{bio-mim-ic-ry bio-in-spi-ra-tion re-us-a-ble pro-vid-er Media-Wiki fuels}

%%%%%%%%%%%%%%%%%%%%%%%% End Helper Commands %%%%%%%%%%%%%%%%%%%%%%%%%%%

%----------------------------------------------------------------------%
% The following is copyright and licensing information for
% redistribution of this LaTeX source code; it also includes a liability
% statement. If this source code is not being redistributed to others,
% it may be omitted. It has no effect on the function of the above code.
%----------------------------------------------------------------------%
% Copyright (c) 2007, 2008, 2009, 2010, 2011 by Theodore P. Pavlic
%
% Unless otherwise expressly stated, this work is licensed under the
% Creative Commons Attribution-Noncommercial 3.0 United States License. To
% view a copy of this license, visit
% http://creativecommons.org/licenses/by-nc/3.0/us/ or send a letter to
% Creative Commons, 171 Second Street, Suite 300, San Francisco,
% California, 94105, USA.
%
% THE SOFTWARE IS PROVIDED "AS IS", WITHOUT WARRANTY OF ANY KIND, EXPRESS
% OR IMPLIED, INCLUDING BUT NOT LIMITED TO THE WARRANTIES OF
% MERCHANTABILITY, FITNESS FOR A PARTICULAR PURPOSE AND NONINFRINGEMENT.
% IN NO EVENT SHALL THE AUTHORS OR COPYRIGHT HOLDERS BE LIABLE FOR ANY
% CLAIM, DAMAGES OR OTHER LIABILITY, WHETHER IN AN ACTION OF CONTRACT,
% TORT OR OTHERWISE, ARISING FROM, OUT OF OR IN CONNECTION WITH THE
% SOFTWARE OR THE USE OR OTHER DEALINGS IN THE SOFTWARE.
%----------------------------------------------------------------------%
