% !TEX program = xelatex
% Based on a template by Jan Küster, licensed under the MIT license
% https://github.com/jankapunkt/latexcv

\documentclass[11pt]{article}
\usepackage{microtype}
\usepackage[default]{raleway}
\usepackage{moresize}

\usepackage[letterpaper]{geometry}

\geometry{top=0.75in, left=0.75in, right=0.75in, bottom=-0.4cm}

\usepackage{fancyhdr}
\pagestyle{fancy}
\cfoot{}

\setlength{\headheight}{-5pt}
\renewcommand{\headrulewidth}{0pt}
\renewcommand{\footrulewidth}{0pt}

\setlength{\parindent}{0pt}

\usepackage{array}
\newcolumntype{x}[1]{%
>{\raggedleft\hspace{0pt}}p{#1}}%

\usepackage{enumitem}
\setlist{nosep, leftmargin=*}

\usepackage{color}
%accent color
\definecolor{sectcol}{RGB}{0,150,255}
%dark background color
\definecolor{bgcol}{RGB}{110,110,110}
%light background / accent color
\definecolor{softcol}{RGB}{225,225,225}

\newcommand{\cvsection}[1]
{
\begin{center}
    \large\textcolor{sectcol}{\textbf{#1}}
\end{center}
}

%----------------------------------------------------------------------------------------
%	 CV EVENT
%----------------------------------------------------------------------------------------

\newcommand{\edevent}[3]
{
\begin{tabular*}{1\textwidth}{p{13.9cm} x{3.04cm}}%
    \textbf{#2} - \textcolor{bgcol}{#3} & \vspace{2.5pt}\textcolor{sectcol}{#1}%
\end{tabular*}

\vspace{-4pt}
}

\newenvironment{cvevent}[3]{%
%
\begin{tabular*}{1\textwidth}{p{13.9cm} x{3.04cm}}%
    \textbf{#2} - \textcolor{bgcol}{#3} & \vspace{2.5pt}\textcolor{sectcol}{#1}%
\end{tabular*}%

\vspace{-8pt}%
\textcolor{softcol}{\hrule}%
\vspace{6pt}%
}{
\vspace{6pt}
}

%----------------------------------------------------------------------------------------
% CUSTOM STRUT FOR EMPTY BOXES
%----------------------------------------- -----------------------------------------------
\newcommand{\mystrut}{\rule[-.3\baselineskip]{0pt}{\baselineskip}}

\newcommand*{\meauthor}[1]{\textbf{#1}}
\newcommand*{\papertitle}[1]{\textit{#1}}

\usepackage{hyperref}
\usepackage{url}
\definecolor{darkblue}{rgb}{0.0,0.0,0.3}
\hypersetup{
    colorlinks,
    breaklinks,
    linkcolor=darkblue,
    urlcolor=darkblue,
    anchorcolor=darkblue,
    citecolor=darkblue,
    pdfinfo={
        Title={Resume of Bryan W. Weber},
        Author={Bryan W. Weber},
    },
}

% You can adjust the style \url{} uses here:
% (options are: same, rm, sf, tt; defaults to tt)
\urlstyle{same}

% For \email{ADDRESS}, links ADDRESS to the url mailto:ADDRESS
% (uncomment to typeset the e\-/mail address in typewriter font;
%  otherwise, will be typeset in the \urlstyle above)
\DeclareUrlCommand\emaillink{\urlstyle{same}}
\providecommand*\emaillink[1]{\nolinkurl{#1}}
\providecommand*\email[1]{\href{mailto:#1}{\emaillink{#1}}}

\usepackage{academicons}
\usepackage{fontawesome}
\newcommand{\iconspace}{\enspace}
\newcommand{\arxiv}[1]{\aiarXiv\iconspace arxiv:\href{https://arxiv.org/abs/#1}{#1}}
\newcommand{\doi}[1]{\aiDoi\iconspace doi:\href{https://doi.org/#1}{#1}}
% \newcommand{\figshare}[1]{\aiFigshare\iconspace figshare:\href{https://doi.org/#1}{#1}}
% \newcommand{\refurl}[1]{\faGlobe\iconspace\url{#1}}

\begin{document}

\pagestyle{fancy}

\vspace{-8pt}
\begin{center}
    {\HUGE \textsc{Bryan W.\ Weber}}%, Ph.D.}}

    \vspace{3pt}

    \email{bryan.w.weber@gmail.com} \enspace\textbullet\enspace \url{https://bryanwweber.com} \enspace\textbullet\enspace \faicon{github} \href{https://github.com/bryanwweber}{bryanwweber} \enspace\textbullet\enspace 412-443-6447
\end{center}

\vspace{-12pt}
\textcolor{softcol}{\hrule}

\normalsize

% \vspace{-6pt}
% \cvsection{Summary}
% Assistant Professor in Residence in Mechanical Engineering with five years experience
% working with diverse groups of students teaching thermal-fluids engineering. Four-time
% recipient of Teaching Commendation for excellence on student evaluations, twice-nominated
% for University Teaching Innovation Award by Department chair. Ph.D.\ focus
% on analyzing large models for combustion of biofuels. Experience with Python, web
% development, and data analysis using Jupyter Notebooks.

%---------------------------------------------------------------------------------------
%	EDUCATION SECTION
%---------------------------------------------------------------------------------------
\cvsection{Education}

\edevent{2014}{Ph.D.}{Mechanical Engineering, University of Connecticut}
\edevent{2010}{M.S.}{Mechanical Engineering, University of Connecticut}
\edevent{2009}{B.S.E.}{Aerospace Engineering, Case Western Reserve University}

%---------------------------------------------------------------------------------------
%	EXPERIENCE
%----------------------------------------------------------------------------------------
\cvsection{Select Experience}

\begin{cvevent}{2014--Present}%
{Assistant Professor in Residence}%
{Mechanical Engr., University of Connecticut}%
\begin{itemize}
    \item Teaching 3 undergraduate courses per semester including strong focus on problem
    solving and quantitative analysis of engineering problems
    \item Recorded over 80 lecture videos for Thermodynamic Principles, available on
    \href{https://www.youtube.com/playlist?list=PLnOxmF4n89SXsKxFb6ug0ThMNpqJST5_X}{YouTube}
    \item Advised 20 senior capstone design student teams on project management, meeting
    deliverables, and adjusting to changing priorities
    % \item Mentoring graduate and undergraduate TAs in grading and tutoring
    % \item Mentored 10 undergraduate and high-school students on software development
    % research projects
\end{itemize}
\end{cvevent}

\begin{cvevent}{2013--Present}
{Open-Source Software Developer}
{\href{https://github.com/bryanwweber/thermostate}{ThermoState},
\href{https://github.com/cantera}{Cantera},
\href{https://github.com/pr-omethe-us/pyked}{ChemKED/PyKED}}
\begin{itemize}
    \item Developing open-source Python software packages for thermodynamics courses,
    modeling of chemical reaction networks, and databases of combustion experiments
    \item Deployed software to classes of over 150 students using Docker containers and
    JupyterHub
    \item Using Jupyter Notebooks to analyze database of 700 combustion experiments to
    build models to predict combustion behavior
    \item Redesigned \url{https://cantera.org} to use modern responsive design and static
    HTML generator
    % \item Organized 4 workshops for approximately 200 Cantera users at regional and
    % national conferences
    \item Posted over 850 messages to support Cantera users on the public
    \href{https://groups.google.com/forum/#!forum/cantera-users}{Google Group}
\end{itemize}
\end{cvevent}

\begin{cvevent}{2009--2014}
{Graduate Research Assistant}
{University of Connecticut}
\begin{itemize}
    \item Developed an open-source Python-based experimental analysis package to quickly
    and repeatably process data files with over 100,000 time-series points
    \item Analyzed kinetic models with thousands of species and tens of thousands of
    reactions to determine the parameters controlling prediction of ignition delay and to
    improve the ability of the models to predict combustion events
\end{itemize}
\end{cvevent}

\cvsection{Select Publications and Presentations}

\meauthor{B.W.\ Weber}. \papertitle{ThermoState: A state manager for thermodynamics
courses}. Journal of Open Source Education, vol.\ 1, no.\ 8, pp.\ 33,
Oct.\ 2018.
\doi{10.21105/jose.00033}

\meauthor{B.W.\ Weber} and K.E.\ Niemeyer. \papertitle{ChemKED: a human- and
machine-readable data standard for chemical kinetics experiments.} International Journal
of Chemical Kinetics, vol.\ 50, no.\ 3, pp.\ 135--148, Mar.\ 2018.
\doi{10.1002/kin.21142} \quad \arxiv{1706.01987v3}

\textbf{Cantera Workshop \& Forum}. Held at 4 National and Regional conferences of the Combustion
Institute. Presented Cantera tutorials \& applications with core developers to over 200
users. \url{https://cantera.github.io/ncm-2019-workshop}

% \meauthor{B.W.\ Weber}, J.A.\ Bunnell, K.\ Kumar, and C.J.\ Sung. \papertitle{Experiments
% and Modeling of the Autoignition of Methyl Pentanoate at Low to Intermediate Temperatures
% and Elevated Pressures in a Rapid Compression Machine.} Fuel, vol.\ 212, pp.\ 479--486,
% Jan.\ 2018. \doi{10.1016/j.fuel.2017.10.037}

\cvsection{Select Awards \& Honors}

\begin{itemize}[leftmargin=*]
    \item \textbf{Four-time} recipient of the University of Connecticut Provost's Teaching
    Commendation, awarded to faculty for excellence on their end-of-semester evaluations
    \item \textbf{Two-time} elected commencement marshal by the senior-class students
    \item \textbf{Two-time} University of Connecticut Teaching Innovation award nominee by
    Department Chair
\end{itemize}

\cvsection{Select Skills}

\begin{tabular}{ll}
\textbf{Programming:} Python, Git, C++, Jupyter, Pandas & \textbf{Web Design:} JavaScript, CSS, Bootstrap v4.1 \\
\textbf{Cloud and DevOps:} Docker, TravisCI, Appveyor &
\textbf{Operating Systems:} Linux, macOS, Windows
\end{tabular}

% \renewcommand\UrlFont{\color{white}}
% \null
% \vspace*{\fill}
% \hspace{-0.25\linewidth}%
% \colorbox{bgcol}{\makebox[1.5\linewidth][c]{\rule[-.3\baselineskip]{0pt}{\baselineskip} \small \url{https://bryanwweber.com} \textbullet\ \url{https://github.com/bryanwweber}}}

\end{document}
